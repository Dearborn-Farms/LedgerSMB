<?lsmb#   This is a comment block; it's ignored by the template engine.

   Version:  1.0
   Date:     2021-01-04
   File:     letterhead.tex
   Set:      demo

Template version numbers are explicitly not aligned across templates or
releases. No explicit versioning was applied before 2021-01-04.

-?>
\parbox{\textwidth}{%
  \parbox[b]{.42\textwidth}{%
    <?lsmb company ?>

    <?lsmb address ?>
  }
  \parbox[b]{.2\textwidth}{
  <?lsmb#  using "comment block" template tag to suppress output

  To include a logo in your LaTeX document, it needs to be retrieved from
  the database and temporarily stored in a location accessible to the PDF
  generator. The function 'dbfile_path' handles that.

  To use this functionality, (a) remove the percent sign before the
  \includegraphics command and (b) make sure all templates which INCLUDE
  'letterhead', contain '\usepackage{graphics}'

  If the generated output is to be PDF output, make sure that the image is
  a PNG image as shown below. In case Postscript (PS) or EPS output, make
  sure the image is (E)PS.

  For information on the avaliable options for the 'includegraphics'
  command, see https://en.wikibooks.org/wiki/LaTeX/Importing_Graphics

  Example:

    \includegraphics[scale=0.3]{<?lsmb dbfile_path('logo.png') ?>}

    ?>
    }\hfill
  \begin{tabular}[b]{rr@{}}
  <?lsmb text('Tel:') ?> & <?lsmb tel ?>\\
  <?lsmb text('Fax:') ?> & <?lsmb fax ?>
  \end{tabular}

  \rule[1.5em]{\textwidth}{0.5pt}
}

